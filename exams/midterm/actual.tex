% Exam Template for UMTYMP and Math Department courses
% http://mathcep.umn.edu/umtymp/
%
% Using Philip Hirschhorn's exam.cls: http://www-math.mit.edu/~psh/#ExamCls
%
% run pdflatex on a finished exam at least three times to do the grading table on front page.
%
%%%%%%%%%%%%%%%%%%%%%%%%%%%%%%%%%%%%%%%%%%%%%%%%%%%%%%%%%%%%%%%%%%%%%%%%%%%%%%%%%%%%%%%%%%%%%%

% These lines can probably stay unchanged, 
\documentclass[addpoints,11pt]{exam}
\RequirePackage{amssymb, amsfonts, amsmath, latexsym, verbatim, xspace, setspace}
% you can remove these two lines if you're not making pictures with tikz.
\RequirePackage{tikz}
\usetikzlibrary {plotmarks}

% By default LaTeX uses large margins.  This doesn't work well on exams; problems
% end up in the "middle" of the page, reducing the amount of space for students
% to work on them.
\usepackage[margin=1in]{geometry}


% Here's where you edit the Class, Exam, Date, etc.
\newcommand{\class}{Math 105A}
\newcommand{\term}{Fall 2017}
\newcommand{\examnum}{Midterm Exam}
\newcommand{\examdate}{Fri Nov 3 2017}
\newcommand{\examtime}{12.00pm}

% For an exam, single spacing is most appropriate
\singlespacing
% \onehalfspacing
% \doublespacing

% For an exam, we generally want to turn off paragraph indentation
\parindent 0ex

\begin{document} 

% These commands set up the running header on the top of the exam pages
\pagestyle{head}
\firstpageheader{}{}{}
\runningheader{\class}{\examnum\ - Page \thepage\ of \numpages}{\examdate}
\runningheadrule

\begin{flushright}
\begin{tabular}{p{2.8in} r l}
\textbf{\class} & \textbf{Student's Name (Print):} & \makebox[1.5in]{\hrulefill}\\
\term & &\\
\examnum &\textbf{Student's ID:} & \makebox[1.5in]{\hrulefill}\\
\examdate &&\\
\examtime &  \textbf{Discussion Section Code:} & \makebox[1.5in]{\hrulefill}
\end{tabular}\\
\end{flushright}
\rule[1ex]{\textwidth}{.1pt}

{\bf Print your name and student ID on the top of this page.} \\

This exam contains \numpages\ pages (including this cover page) and
\numquestions\ problems. 
% {\bf Note that some equations are numbered.}  
You may \textit{not} use your books, notes, or any calculator in this exam. Do not write in the grading table below.\\

The following rules apply to the answers you provide in this exam:

\begin{minipage}[t]{3.7in}
\vspace{0pt}
\begin{itemize}

\item \textbf{Organize your work}, in a neat and coherent way.   

\item \textbf{Unsupported answers will not receive full credit}.  Calculation or verbal explanation is expected. 

\item \textbf{If you need more space, use the back of the pages}; clearly indicate when you have done this.

\item \textbf{Box your final answer} for full credit. 

\end{itemize}
\end{minipage}
\hfill
\begin{minipage}[t]{2.3in}
\vspace{0pt}
\gradetable[v]%[pages]  % Use [pages] to have grading table by page instead of question
\end{minipage}

\newpage % End of cover page

%%%%%%%%%%%%%%%%%%%%%%%%%%%%%%%%%%%%%%%%%%%%%%%%%%%%%%%%%%%%%%%%%%%%%%%%%%%%%%%%%%%%%
%
% See http://www-math.mit.edu/~psh/#ExamCls for full documentation, but the questions
% below give an idea of how to write questions [with parts] and have the points
% tracked automatically on the cover page.
%
%
%%%%%%%%%%%%%%%%%%%%%%%%%%%%%%%%%%%%%%%%%%%%%%%%%%%%%%%%%%%%%%%%%%%%%%%%%%%%%%%%%%%%%

\begin{questions} 

\question
\begin{parts} 
\part[5]
Consider the sequence $\{ p_n \}$ defined by 
\[
p_n = \sum_{k=1}^n \frac{1}{k}.
\]
This sequence diverges (to see this intuitively note that $p_n \approx \int_1^\infty dx /x$ for large $n$). 
Show that, nevertheless, 
\[
\lim _{n\rightarrow \infty} (p_n - p_{n-1}) = 0. 
\]
Less: Be mindful of this when testing for convergence in your iterative algorithms! 

\vfill 

\part[5]
Show that $p_n = 10^{-2^n} $ converges quadratically. 
\vfill 
\end{parts} 

\newpage 

%\question
%In class, we proved that if $g(x)$ maps an interval onto itself, and $|g'(x)| < 1$ in that interval, then $g$ has  exactly one fixed point there. 
%\begin{parts} 
%\part[5]
%Show that the theorem's assumptions are true for $g(x) = 1 + \sqrt{x}$. 
%\vfill
%\part[10]
%Find the unique fixed point. 
%\vfill
%\end{parts} 

\question[10]
Find the unique fixed point of $g(x) = 1 + \sqrt{x}$. 

\vfill 
\vfill 

%\question[5]
%Use Horner's method to compute $P(3)$ where 
%\[
%P(x) = 2x^3 - 6x^2 + 2x - 1.
%\]
%
%\vfill 


\question[5] 
Find all values of $\alpha$ for which the following system has no solutions. Explain your answer. 
\begin{eqnarray*} 
2x  -y + 3z & = & 5 \\
4y - 4z & = & -4 \\
(5 + \alpha)z & = & 8 + \alpha 
\end{eqnarray*} 

\vfill 

\newpage 

\question[10] 
Use Gaussian Elimination to compute the determinant of 
\[ 
A = \left[
\begin{array}{ccc} 
1 & 1 & 1 \\
0 & 1 & 1 \\
1 & 2 & 1
\end{array}
\right] .
\]

\newpage 

\question[20]
\label{PLU_question} 

Use partial pivoting to compute a $PLU$ factorization of 
\[ 
A = 
\left[
\begin{array}{ccc} 
0 & 1 & 1 \\
1 & 1 & 1 \\
-1 & 1 & -1
\end{array}
\right] .
\]

\newpage 
(You may use this page to complete your solution to Q\ref{PLU_question}.) 


\end{questions} 

\end{document}
